%% LyX 2.1.4 created this file.  For more info, see http://www.lyx.org/.
%% Do not edit unless you really know what you are doing.
\documentclass[twocolumn,conference]{IEEEtran}
\usepackage[T1]{fontenc}
\usepackage{float}
\usepackage{amsmath}
\usepackage{graphicx}
\usepackage[unicode=true,
 bookmarks=true,bookmarksnumbered=true,bookmarksopen=true,bookmarksopenlevel=1,
 breaklinks=false,pdfborder={0 0 0},backref=false,colorlinks=false]
 {hyperref}
\hypersetup{pdftitle={Your Title},
 pdfauthor={Your Name},
 pdfpagelayout=OneColumn, pdfnewwindow=true, pdfstartview=XYZ, plainpages=false}

\makeatletter

%%%%%%%%%%%%%%%%%%%%%%%%%%%%%% LyX specific LaTeX commands.
%% A simple dot to overcome graphicx limitations
\newcommand{\lyxdot}{.}


%%%%%%%%%%%%%%%%%%%%%%%%%%%%%% User specified LaTeX commands.
% for subfigures/subtables
\usepackage[caption=false,font=footnotesize]{subfig}

\makeatother

\begin{document}





\title{Exploring metrics of analyzing citation networks }


\author{\IEEEauthorblockN{Suhas~Gundimeda}\IEEEauthorblockA{Indian Institute
of Technology Madras}\and \IEEEauthorblockN{Karnaty~Rohit Reddy}\IEEEauthorblockA{Indian
Institute of Technology Madras}}


\IEEEspecialpapernotice{Invited Paper}


\IEEEaftertitletext{after title text like dedication}
\maketitle
\begin{abstract}
We describe some measures and metrics proposed by various papers in
scientometrics to analyze citation and co-citation networks.
\end{abstract}


\IEEEpeerreviewmaketitle{}


\section{Introduction}

Scientometrics deals with the measuring and analysing science, technology
and innovation.  In this paper we look into analysis of scientific
contributions in peer-reviewed publications using their citation,
co-citation and co-authorship networks.


\section{Previous Work}






\subsection{Citation count}

Citation count is the sum of all published citations to a scientific
paper. This is the most widely used measure of impact, and is the
standard metric against which all of the others are compared.


\subsection{PaperRank}

The PageRank algorithm has the problem of nodes with no inbound edges
causing it to have no solution. To solve this problem the algorithm
is modified to assign rank of 1 to all initial nodes and reset it
to zero after computation.\cite{key-2}


\subsection{h-index}

An author metric whose value is the maximal number $h$ such that
the author has at least $h$ papers with at least $h$ citations.

\begin{figure}[H]
\centering{}\includegraphics[scale=0.9]{hindex}\caption{h-index illustrated}
\end{figure}



\subsection{PaperRank-Hirsch index}

This is a proposed alternative to h-index which uses the calculated
PaperRanks. The PaperRanks are normalized to represent more meaningful
quantities, namely the most probable number of citations which the
node might receive, the $Q$-value. The index is then defined as the
integer number $h$ such that an author has at least $h$ papers with
$Q$-value greater than or equal to $h$.\cite{key-2}


\subsection{CoRank}



The CoRank algorithm\cite{key-1}, shown by equation, is based upon
the principals of the PageRank algorithm, as such it is simply a variation
of PageRank algorithm where the input data has changed rather than
the algorithm itself. The key difference is that in CoRank, the rank
gained from each co-cited publication is the CoRank of that paper
CR(p$_{j}$) divided by the number of co-citations (Co-Links) CL(p$_{j}$)
that paper has with other papers

\begin{center}
$CR(n)=\frac{1-\alpha}{|V|}+\alpha\underset{p_{j}\in M(p_{i})}{\Sigma}\frac{CR(p_{j})}{CL(p_{j})}$
\par\end{center}


\subsubsection{CoRank-LinkCount}

CoRank-LinkCount is an algorithm without a weighting factor, meaning
that iterative calculation is not required. CoRank-LinkCount looks
solely at the number of citations which are received, not by the publication
in question, but by all the publications with which the subject paper
is co-cited.

Figure 2 shows how the sum of co-citing publications is worked out.
Publication n represents the target publication and each p being the
co-cited publication; each publication marked c is thus a directly
citing publication to n. The total CoRankLinkCount is the sum of all
citations towards those papers (p) a publication (n) is co-cited with,
including duplicates and c itself

\begin{figure}[H]
\centering{}\includegraphics[scale=0.7]{corank-linkcount}\caption{Network of Co-Cited publications from publication n\cite{key-1}}
\end{figure}



\subsection{Betweeness centrality}

\begin{figure}[H]
\includegraphics[scale=0.4]{600px-Graph_betweenness\lyxdot svg}

\caption{Hue (from red = 0 to blue = max) shows the node betweenness.}


\end{figure}


Betweenness centrality is based on the number of shortest paths passing
through a vertex. Vertices with a high betweenness play the role of
connecting different groups. In the following formula, g$_{jik}$
is all geodesics linking node j and node k which pass through node
i. g$_{jk}$ is the geodesic distance between the vertices of j and
k. 

\begin{center}
$C_{B}(n_{i})=\underset{j,k\neq i}{\Sigma}\frac{g_{jik}}{g_{jk}}$ 
\par\end{center}

In social networks, vertices with high betweenness are the brokers
and connectors who bring others together \cite{key-4}. Being between
means that a vertex has the ability to control the flow of knowledge
between most others. Individuals with high betweenness are the pivots
in the network knowledge flowing. The vertices with highest betweenness
also result in the largest increase in typical distance between others
when they are removed.\cite{key-3}


\subsection{Closeness centrality on co-authorship networks}

Closeness centrality \cite{key-5} emphasizes the distance of a vertex
to all others in the network through the geodesic distance from each
vertex to all others. It is a measure of how long it will take for
information to spread from a particular node \cite{key-4}.

\[
C_{c}(n_{t})=\sum_{i=1}^{N}\frac{1}{d(n_{i},n_{j})}
\]


Where $d(n_{i},n_{j})$ is the distance from node $i$ to node $j$
and $C_{c}$ is the closeness centrality.\cite{key-3}


\subsection{Group centrality measures}

According to \cite{key-7}, the degree centrality of a group is the
number of actors outside the group that are connected to the member
of that group. Different ties to the same actors by different group
members are only counted once. The formula is described below: 

\begin{eqnarray*}
Group\,degree\,centrality & = & |N(C)|\\
Normalized\,group\,degree\,centrality & = & \frac{|N(C)|}{|V|-|C|}
\end{eqnarray*}
Where C is a group that is a subset of the set of vertices V, and
N(C) denotes the set of all nodes that are not in C but connected
to a member of C.

Similarly group closeness and betweeness centralities are explored
in \cite{key-6,key-7}. These measures help us analyse the interdisciplinary
nature of disciplines. 
\begin{thebibliography}{1}
\bibitem{key-1}DAVID C TARRANT (2011) \textquotedbl{}A Study of Early
Indication Citation Metrics \textquotedbl{}, University of Southampton,
Electronics and Computer Science, PhD Thesis.

\bibitem{key-2}Mikalai Krapivin, Maurizio Marchese and Fabio Casati
''Exploring and Understanding Scientific Metrics in Citation Networks''

\bibitem{key-3}Erjia Yan, Ying Ding,\href{http://arxiv.org/pdf/1012.4862.pdf}{ \textquotedblleft Applying centrality measures to impact analysis: A coauthorship network analysis\textquotedblright} 

\bibitem{key-4}Yin, L., Kretschmer, H., Hanneman, R. A., \& Liu,
Z. (2006). Connection and stratification in research collaboration:
An analysis of the COLLNET network. Information Processing and Management,
42 , 1599- 1613

\bibitem{key-5}Freeman, L.C. (1979). Centra lity in social networks.
Conceptual clarification. Social Networks, 1 , 215-239

\bibitem{key-6}Chaoqun Ni, Cassidy R. Sugimoto, and Jiepu Jiang.
Degree, Closeness, and Betweenness: Application of group centrality
measurements to explore macro - disciplinary evolution diachronically

\bibitem{key-7}Everett, M. G., \& Borgatti, S. P. (20 05). Extending
centrality. In P. J. Carrington, J. Scott \& S. Wasserman (Eds.),
Models and methods in social network analysis (pp. 57 - 76). New York:
Cambridge University Press.

\end{thebibliography}
$ $
\end{document}
